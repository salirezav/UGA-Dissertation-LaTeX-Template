% !TEX root =../dissertation.tex
\documentclass[./dissertation.tex]{subfiles}
\begin{document}
\chapter{Toward a Foundation Model for Biomedical Image Segmentation}

\section{Introduction}

% 1. Supervised methods require large annotated data to perform well. Their generalizability suffers. \\
% 2. Unsupervised methods are good in that they don't need a lot of annotated data, but their accuracy is not as good and they can be very domain-specific. \\

Biomedical image segmentation is pivotal across a diverse array of medical and biological applications, from diagnostic imaging to cellular analysis. While supervised segmentation methods, especially convolutional neural networks (CNNs), have achieved remarkable accuracy, their reliance on large, annotated datasets severely limits their generalizability and transferability, particularly in biomedical contexts where labeled data is scarce and costly to obtain. Conversely, unsupervised methods, despite being simpler and more generalizable, often fall short in segmentation precision and robustness, thus failing to meet the accuracy requirements of clinical and research settings.


% 3. foundation models are trained on very large datasets. SAMs are good at zero-shot segmentation for they learn the repesentation of textures, objects, etc. they can take in prompts in the form of bounding boxes and points. these SAMs can potentially be a good easy to use choice for producing good-enough masks. \\
% 4. How SAMs work. the internal mechanism that allows it to take points and generage masks. or just generate masks for all objects in the image. \\
% 5. SAM struggles with biomedical images since it is originally trained on general images (e.g. horse, cat, cat, etc.). that's why it needs to be fine-tuned to be able to perform reliably for biomedical images. \\

Recently, the emergence of Foundation Models (FMs) and the Segment Anything Model (SAM) offers a compelling new direction for addressing these limitations. SAM, introduced by Kirillov et al. \cite{kirillov2023segment}, marked a significant advancement by achieving impressive zero-shot segmentation capabilities across varied image domains, relying only on minimal user prompts such as points or bounding boxes. Its successor, SAM 2, further extends these capabilities into video segmentation, leveraging advanced architectures to improve accuracy and interaction efficiency \cite{ravi2024sam}. These models are trained on large and diverse datasets, enabling them to generalize effectively across multiple segmentation tasks without extensive domain-specific training. Despite their potential, the application of general-purpose SAM models to biomedical imaging presents unique challenges. The complex and nuanced nature of biomedical images characterized by varying imaging modalities, structures, textures, and noise levels means that models trained primarily on general-domain images may struggle to achieve desirable segmentation accuracy \cite{mazurowski2023segment,na2024segment}. Recognizing this, adaptations of SAM tailored specifically to biomedical contexts have emerged. Models such as MedSAM \cite{ma2024segment} and MediViSTA-SAM \cite{kim2025medivista} demonstrate the feasibility of adapting SAM to medical imaging and video analysis, showing promising results that often surpass specialized, modality-specific models in robustness and accuracy.



% 6. There are many biomedical adaptations for SAM, including this and that and the other one which do this and that and something else, each of which focusing on improving some aspect of SAM. They all show that their work has promising improvements. \\

% 7. There's another set of research done on incorporating text prompts into SAM. it involves training text-image pairs first and the model uses contrastive learning to clusterize or localize the similar text-image embeddings. CLIP is a famous model. in inferrence, this CLIP model takes the prompt and image and tries to highlight the locations on the image which it finds relevant to the prompt. then some random points or bounding boxes are created using this heatmap and then passed to SAM for downstream segmentation. there are of course some biomedical adaptations of CLIP like BiomedCLIP. they perform great vs. the vanilla CLIP. \\


% 8. But there is so much stochasticity and randomness within this process. from the way the CLIP model is trained or fine-tuned to how this inference is done and how the point and box prompts are selected. they all affect the final outcome of the SAM's mask decoder. \\

Alongside spatial segmentation capabilities, the integration of vision-language models like BiomedCLIP into SAM workflows has gained considerable attention. BiomedCLIP, trained on extensive biomedical image-text pairs, provides robust multimodal embeddings that bridge textual and visual domains, enabling powerful text-driven segmentation. Frameworks such as MedCLIP-SAMv2 exemplify this integration, demonstrating how textual prompts can effectively guide precise segmentation tasks, from identifying tumors in medical scans to delineating specific cellular structures in microscopy images. However, the integration of vision-language models with SAM introduces additional layers of complexity and stochasticity. Variability in outputs stemming from factors such as the fine-tuning process of models like BiomedCLIP, differences in textual prompts, or randomness in inference strategies can undermine reproducibility, a critical factor in biomedical research and clinical applications. Thus, understanding and mitigating this stochasticity is paramount for the practical adoption and reliability of these models.

% 9. that's why we're focusing on this part, aiming to find ways to control this randomness and increase the reproducibility of the CLIP, hence helping SAM in honing in on the ROI that best matches the prompt.

This chapter aims to comprehensively explore and address these challenges. We focus specifically on evaluating and enhancing the reproducibility of integrated vision-language segmentation models, investigating how different factors—including fine-tuning strategies, prompt engineering, and inference methodologies—influence variability in segmentation outcomes. Our goal is to develop methodologies that standardize and optimize these variables, ensuring SAM-based models are not only accurate and versatile but also reliably reproducible in biomedical contexts.

\section{Background}

% 1. SAM's original paper, and the subsequent SAM2. \\

SAM \cite{kirillov2023segment} introduced a groundbreaking promptable segmentation approach, trained on a vast dataset (SA-1B), featuring over one billion masks. Its architecture comprises three components: a powerful Vision Transformer (ViT) image encoder, a flexible prompt encoder handling points, boxes, and text, and a lightweight mask decoder to produce segmentation masks. SAM's notable innovation lies in its zero-shot capabilities—achieved through prompt engineering, enabling it to generalize well across various segmentation tasks, even without task-specific fine-tuning. SAM-2 \cite{ravi2024sam} extends SAM's capabilities to video data, incorporating a memory attention mechanism that retains information from previous frames, thereby significantly enhancing segmentation accuracy and interaction efficiency. SAM-2 utilizes a hierarchical transformer architecture (Hiera) \cite{ryali2023hiera,bolya2023window} pre-trained with masked autoencoders (MAE) \cite{he2022masked}, making it highly effective for real-time segmentation tasks across images and videos. This memory-enhanced architecture allows SAM-2 to iteratively refine masks, leading to considerable improvements in temporal segmentation consistency

% 2. Biomedical adaptations of SAM and what they focus on. \\

The success of the Segment Anything Model (SAM) in general-domain image segmentation has inspired adaptations and methodological enhancements aimed at tailoring its capabilities specifically for biomedical applications, addressing unique challenges associated with medical image segmentation. MedSAM \cite{ma2024segment} and other models, like BioSAM-2 \cite{yan2024biomedical}, have demonstrated the necessity of domain-specific fine-tuning for achieving clinical-grade accuracy. BioSAM-2, particularly designed for biomedical segmentation, has optimized SAM-2 with medical domain-specific data and additional memory mechanisms for improved performance across diverse biomedical imaging modalities. Medical SAM Adapter (Med-SA) \cite{wu2023medical} introduced adaptation modules such as Space-Depth Transpose (SD-Trans) and Hyper-Prompting Adapter (HyP-Adpt), which enhance SAM's performance on medical images through minimal yet strategic parameter adjustments. These modules have shown significant improvements over traditional segmentation methods by efficiently incorporating

In the realm of prompt learning and auto-prompting the Segment Any Cell (SAC) \cite{na2024segment} framework leveraged auto-prompting and fine-tuning methods, using Low-Rank Adaptation (LoRA) \cite{hu2022lora}, to automatically generate effective prompts for nuclei segmentation. This method reduced manual intervention and improved segmentation accuracy in microscopic imaging scenarios. SSPrompt \cite{huang2024learning} optimized SAM's spatial and semantic prompts directly within its embedding space, enhancing its generalization capabilities across complex segmentation tasks. The Segment and Caption Anything \cite{huang2024segment} model enriched SAM's semantic understanding capabilities by integrating a query-based feature mixer, improving semantic precision and enabling the model to provide meaningful regional captions, thus enhancing segmentation results through better semantic contextualization.

A comprehensive survey titled "Foundation Models for Biomedical Image Segmentation" \cite{lee2024foundation} underscores the transformative potential of SAM, summarizing over 100 studies that have successfully adapted SAM to a wide range of biomedical datasets. The survey highlights SAM's strong zero-shot capabilities and outlines various domain-specific tuning methods and data scarcity challenges that have driven innovation in biomedical segmentation.


% 3. Biomedical adaptations of CLIP. some researches matched the best description to the image. some other did other things. \\

Incorporation and Integration of Text Prompts in Biomedical SAM
% a. Vision-Language Models and Multi-Modal Integration
Models like BiomedCLIP and adaptations such as MedCLIP-SAMv2 underscore the importance of text-driven segmentation approaches, leveraging extensive biomedical image-text pairs to provide robust multimodal embeddings. This integration enables powerful and precise segmentation guided by textual descriptions, thus bridging visual and textual biomedical data effectively.

The EVF-SAM \cite{zhang2024evf} model exemplifies the integration of early vision-language fusion. It incorporates an early fusion mechanism, significantly outperforming late fusion models by enhancing text-to-image attention, which is critical for accurate segmentation guided by referring expressions.

% b. Specific Applications and Advancements in Text-guided Biomedical Segmentation
Polyp-SAM++ \cite{biswas2023polyp} demonstrated the effectiveness of detailed textual prompts specifically for colorectal polyp segmentation, showing how text guidance could substantially improve the segmentation accuracy and robustness of SAM, particularly in clinically relevant contexts . Hi-SAM \cite{ye2024hi} extended SAM’s capabilities to hierarchical text segmentation, including pixel-level text, word, text-line, and paragraph segmentation, thus enabling more structured and detailed biomedical image analyses, crucial for applications like pathology slide examination. PROMISE \cite{li2024promise} and similar models have adapted SAM to 3D biomedical segmentation, introducing lightweight adapters for depth-related spatial context and achieving superior performance in tumor segmentation tasks by effectively combining textual prompts with depth-awareness.

% 4. Universal Segmentation and Multi-Modal Integration for Enhanced Biomedical Analysis
The Segment Anything with Text prompts (SAT) \cite{zhao2023one} model, trained on an extensive dataset comprising over 22,000 medical scans and nearly 500 anatomical classes, exemplifies a universal segmentation framework that integrates extensive medical terminologies as textual prompts. This approach emphasizes the utility of incorporating domain-specific knowledge directly into the model training, significantly improving segmentation performance across diverse medical imaging tasks.


\section{Methodology}
1. my methodology involves fine-tuning BiomedCLIP on the dataset of cilia i have. the cilia dataset contains videos or nasal epithelial biopsy containing ciliary regions. these videos are annotated and the annotations show where the body of the cell is and where the ciliary region are. sometimes the ciliary region sticks out of the boundary of the cell which makes them easy to detect with eye and some other times they overlap the body of the cell or are out of focus which makes them very hard to detect even by human. anyway. I finetune the BiomedCLIP model using masked images of cell bodies and masked images of cilia along with their respective textual descriptions and then use the model to infer on unseen data. my aim is to find out which factors undermine the reproducibility of this process. those are: fine-tuning the model and for how long and how much, the text descriptions of the patches, whether using masked images or raw images, and - finally when the BiomedCLIP spits out a heatmap of its predictions - what's the best strategy in selecting positive/negative points or bounding boxes considering the output of the BiomedCLIP - that is going to be the input of SAM - to make sure we get accurate and reproducible results

\section{Results and Discussion}

\section{Conclusion and Final Remarks}


\end{document}